\documentclass{article}
\usepackage[utf8]{inputenc}
\usepackage{amsmath}
\usepackage{amsfonts}
\usepackage{amssymb}
\usepackage{array}

\usepackage{float}
\restylefloat{table}


%fancy headers
\usepackage{fancyhdr}
\pagestyle{fancy}
\fancyhf{}
\lhead{UAlbany IEEE Meeting Agenda}
\rhead{\thepage}


\author{James Oswald}
\title{UAlbany IEEE Fall 2020 \\ First Officer Meeting Agenda}


\begin{document}
\maketitle
\thispagestyle{fancy}
\section{Meetings}
\subsection{Meeting Schedule}
\subsubsection{Preliminary Issues}
The challenge posed by scheduling meetings with most people on-line is rather large.
While timezones won't be an issue for most students located in the US, we have two E-Board Members
in South Korea. For reference KST is a full 13 hours ahead of EST.

\subsubsection{Time Frames}
In the past the IEEE has been known to have two general time frames for meetings open to students, an afternoon time frame and a nighttime time frame. 
\begin{table}[H]
\begin{tabular}{| m{2.3cm} | m{2.2cm} | m{2.2cm} | m{6cm} |}
\hline
\textbf{Time Frame Name} & \textbf{EST time start} & \textbf{KST time start} & \textbf{Typical Usage of Time} \\
\hline
Afternoon Meeting & 12pm - 2pm & 1am - 3am & General Meetings, Workshops, \par Company / Research Talks. \\
\hline
Nighttime Meeting & 6pm - 8pm & 7am - 9am & Workshops, General Meetings, \par Officer Meetings\\
\hline
Unholy \par Discord Call & 1am - 2am & 2pm - 3pm & Officer Meetings\\
\hline
\end{tabular}
\end{table}
As can be seen, both the afternoon meetings and nighttime meetings are both hard to meet in KST and will need to be scheduled with these constraints in mind.
\vspace{2mm}\newline
\textbf{Question:} What meeting type works best for everyone?
\vspace{2mm}\newline
\textbf{Unfortunate Reality:} Scheduling right now is brutal and it can't be guaranteed we can get a time that works for everyone, and we may need to start prioritizing meetings for most members.

\newpage
\subsubsection{Meeting Types}

A normal season of IEEE would typically compose the following meeting types:
\begin{enumerate}
    \item \textbf{General Meetings:} A laid back meeting that provides an environment to talk with fellow club members, receive help with homework, discuss technology and let eboard chat about administrative matters.
    \item \textbf{Workshops:} A topic driven meeting at which a presentation is presented on a specific topic with potential interactive elements. Can also be fully interactive meeting like interview prep.
    \item \textbf{Talks:} A researcher / company representative / member comes to discuss their work while attendees listen, lack of interactive events. 
    \item \textbf{Officer Meetings:}  An opportunity for the Eboard and JBoard to discuss administrative matters and work on workshops.   
\end{enumerate}
\vspace{2mm}\newline
\textbf{Proposal:} We cut general meetings and move to a workshop / talk exclusive schedule. Due to our growing discord presence and community, we can use this space to take the place of general meetings for the time being this semester. Officer meetings should still be held weekly or bi-weekly 
\vspace{2mm}\newline
\textbf{Question:} weekly or bi-weekly officer meetings?

\subsubsection{Meeting Days}
We have a few issues to consider when picking a week day for workshops / talks. Firstly, Ideally I would like it not to overlap with the CDO meeting days as we're trying to build a good relationship with them and get free advertising. On top of this we need to worry about workshops. 
\vspace{2mm}\newline
\textbf{Question:} Weekly or bi-weekly workshops or a mix? We will have at least 5 and at most 10.

\paragraph{}
One option we have is the \textbf{Noor Quarantine Model (NQM)} from last semester, under which we have bi-weekly workshops on Saturday afternoons with an impromptu unholy discord call officer meeting on Friday nights. Potential issues with this include the fact that afternoon EST meetings are bad for our KST members. However we could alter this to have the workshops be at any time.

\paragraph{}
Another option we have is the \textbf{Night time Round Robin Model (NRRM)} This model shifts the day of the meeting forward by one day every meeting (To simplify: have a workshop every 15 days instead of 14). This model works to reduce scheduling conflicts by allowing people who may have something one or two days of the week to attend the maximum possible number of workshops / talks. Potential issues include lack of a set day for people to remember.
\vspace{2mm}\newline
\textbf{Question:} Which option do you prefer? Is there a 3rd even better option? 

\newpage

\subsubsection{Goals}
By the end of this subsection we need to have agreed on the following:
\begin{enumerate}
    \item Workshop / Talk days
    \item Workshop / Talk times
    \item Number of Workshops / Talks
    \item Officer Meeting days
    \item Officer Meeting times
\end{enumerate}

\subsection{Meeting Content}

\subsubsection{Workshops}
This semester we will be having at least 5 - 10 workshops (some of these may be replaced with talks)(By this point we should have decided on the exact number). We need to decide on content and who will present. Here are the four categories of potential workshops in order from hardest to easiest:
\begin{enumerate}
    \item \textbf{New workshops} Workshops designed from scratch on a topic, these are the most desirable kind of workshop and I hope we all can do at least one.
    \item \textbf{Reused workshops} Workshops that already exist from last semester / have templates online. This consists of memorizing and giving a presentation on material you're familiar with. 
    \item \textbf{Hosting a coding interview} You bring problems from a coding interview prep site online, ask members to solve them and critique their solution and performance.
    \item \textbf{Host a technical talk watch party and discussion} This requires almost no effort, you host a shared stream of an online technical talk on a topic then discuss it with attendees. 
\end{enumerate}
\vspace{2mm}\newline
Here is a list of workshops we have that currently exist from last semester and may be reusable, however if possible I would prefer to avoid reusing them, despite the fact that they had low attendance numbers. 
\begin{enumerate}
    \item \textbf{Esoteric Programming Languages} A 42 page presentation on esolangs, focuses on brainfuck and jsfuck
    \item \textbf{Mastering Remote Management with Linux Servers} A 35 slide introduction to remote management tools available for linux servers and how to use them. 
    \item \textbf{Hosting Your Own Web Portfolio:} An in depth 37 slide presentation about web servers, how they work, and how you can create a profile on one. 
\end{enumerate}

\newpage
\paragraph{}
I had originally intended to write up a schedule of related workshops to be given in order but feared this might be too restrictive, and wanted you to be able to have more creative control of what was being presented.
\vspace{2mm}\newline
\textbf{Proposal:} I would like everyone on the E-Board to create at least one new workshop either this or next semester. Creating and giving presentations is an important skill set and I hope having this as a requirement will help you further your skills. If you really don't feel comfortable you can opt for one of the easier workshops or opt-out entirely if you have a good reason.
\vspace{2mm}\newline
\textbf{Question:} How will we divide the number of workshops? How many are you personally willing to do? What types will you be doing? 





\subsubsection{Talks}

We would like to do virtual talks this semester but I have yet to find or search for anyone willing to give a talk. Professor Muckell mentioned in an email that he would be willing to open up his LLC talks to us but this has yet to be formalized.
\vspace{2mm}\newline
\textbf{Question:} Do you have any connections to anyone who could give a talk? Could You Give a Talk?

\subsubsection{Goals}
By the end of this subsection we need to have agreed on the following:
\begin{enumerate}
    \item How many workshops everyone will be doing
    \item What types of workshops everyone will be doing
    \item Who's trying to get us talks
\end{enumerate}
\section{Other things}
\subsection{SA Junk}
We need to figure out what to do about the SA, I'm pretty sure James and Josh need to take the treasurer exam. We also need to make a proposal for funding. 
\vspace{2mm}\newline
\textbf{Question:} Does anyone want to be in charge of SA related affairs? 
\vspace{2mm}\newline
\textbf{Activity:} Figure out what we need to do for the SA this semester, all of it.

\subsection{The General Meeting}
We need to figure out what day and what time to have the general interest meeting, personally I thought it would make sense to have it be on the workshop schedule for next weeks time slot. On top of this, we need to make a presentation and come up with what to talk about.  
\vspace{2mm}\newline
\textbf{Proposal:} We get Noor's presentation from last semester and use it as a base for this semester's presentation. 


\subsection{Goals}
By the end of this subsection we need to have agreed on the following:
\begin{enumerate}
    \item What needs to be done for the SA.
    \item When we're hosting the general meeting.
    \item What will be presented and who will be presenting at the general intrest meeting.  
\end{enumerate}

\end{document}

